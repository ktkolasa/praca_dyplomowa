\documentclass[a4paper,12pt,twoside,openany]{report}
%
% Wzorzec pracy dyplomowej
% J. Starzynski (jstar@iem.pw.edu.pl) na podstawie pracy dyplomowej
% mgr. inż. Błażeja Wincenciaka
% Wersja 0.1 - 8 października 2016
%
\usepackage{polski}
\usepackage{helvet}
\usepackage[T1]{fontenc}
\usepackage{anyfontsize}
\usepackage[utf8]{inputenc}
\usepackage[pdftex]{graphicx}
\usepackage{tabularx}
\usepackage{array}
\usepackage[polish]{babel}
\usepackage{subfigure}
\usepackage{amsfonts}
\usepackage{verbatim}
\usepackage{indentfirst}
\usepackage[pdftex]{hyperref}


% rozmaite polecenia pomocnicze
% gdzie rysunki?
\newcommand{\ImgPath}{.}

% oznaczenie rzeczy do zrobienia/poprawienia
\newcommand{\TODO}{\textbf{TODO}}


% wyroznienie slow kluczowych
\newcommand{\tech}{\texttt}

% na oprawe (1.0cm - 0.7cm)*2 = 0.6cm
% na oprawe (1.1cm - 0.7cm)*2 = 0.8cm
%  oddsidemargin lewy margines na nieparzystych stronach
% evensidemargin lewy margines na parzystych stronach
\def\oprawa{1.05cm}
\addtolength{\oddsidemargin}{\oprawa}
\addtolength{\evensidemargin}{-\oprawa}

% table span multirows
\usepackage{multirow}
\usepackage{enumitem}	% enumitem.pdf
\setlist{listparindent=\parindent, parsep=\parskip} % potrzebuje enumitem

%%%%%%%%%%%%%%% Dodatkowe Pakiety %%%%%%%%%%%%%%%%%
\usepackage{prmag2017}   % definiuje komendy opieku,nrindeksu, rodzaj pracy, ...


%%%%%%%%%%%%%%% Strona Tytułowa %%%%%%%%%%%%%%%%%
% To trzeba wypelnic swoimi danymi
\title{Wycena działek budowlanych z wykorzystaniem uczenia maszynowego}

% autor
\author{Katarzyna Kolasa}
\nrindeksu{284919}

\opiekun{dr hab. inż.  Paweł Piotrowski}
\terminwykonania{1 lutego 2022} % data na oświadczeniu o samodzielności
\rok{2022}


% Podziekowanie - opcjonalne
%\podziekowania{\input{podziekowania.tex}}

% To sa domyslne wartosci
% - mozna je zmienic, jesli praca jest pisana gdzie indziej niz w ZETiIS
% - mozna je wyrzucic jesli praca jest pisana w ZETiIS
\miasto{Warszawa}
\uczelnia{POLITECHNIKA WARSZAWSKA}
\wydzial{WYDZIAŁ ELEKTRYCZNY}
\instytut{INSTYTUT ELEKTROENERGETYKI}
%\instytut{INSTYTUT ELEKTROTECHNIKI TEORETYCZNEJ\linebreak[1] I~SYSTEMÓW INFORMACYJNO-POMIAROWYCH}
\zaklad{ZAKŁAD SIECI I SYSTEMÓW ELEKTROENERGETYCZNYCH}
% \zaklad{ZAKŁAD ELEKTROTECHNIKI TEORETYCZNEJ\linebreak[1] I~INFORMATYKI STOSOWANEJ}
%\kierunekstudiow{INFORMATYKA}

% domyslnie praca jest inzynierska, ale po odkomentowaniu ponizszej linii zrobi sie magisterska
%\pracamagisterska
%%% koniec od P.W

\opinie{%
  \input{opiniaopiekuna.tex}
  \newpage
  \input{recenzja.tex}
}

\streszczenia{
  \input{streszczenia.tex}
}

\begin{document}
\maketitle

\chapter{Wstęp do problemu wyceny działek budowlanych}
Problem wyceny dziełek budowlanych związany jest z zagadnieniem wyceny nieruchomości w ogóle. 

\section{Definicja działki budowlanej}
 definiuje działkę budowlaną jako
\textit{(..) nieruchomość gruntową lub działkę gruntu, której wielkość, cechy geometryczne, dostęp do drogi publicznej oraz wyposażenie w urządzenia infrastruktury technicznej spełniają wymogi realizacji obiektów budowlanych wynikające z rozporządzenia, odrębnych przepisów i aktów prawa miejscowego}\cite{OMI}


--goły tekst obwieszczenia:
 Działka budowlana przewidziana pod zabudowę budynkami przeznaczonymi na pobyt ludzi powinna mieć zapewnioną możliwość przyłączenia uzbrojenia działki lub bezpośrednio budynku do sieci wodociągowej, kanalizacyjnej, elektroenergetycznej i ciepłowniczej, a dla budynków wymienionych w § 56 – także telekomunikacyjnej. 2. Za równorzędne z przyłączeniem do sieci elektroenergetycznej i ciepłowniczej uznaje się zapewnienie możliwości korzystania  z indywidualnych  źródeł  energii  elektrycznej  i ciepła,  odpowiadających  przepisom  odręb
----
Tymaczesm Ustawa o gospodarce nieruchomościami  definiuje działkę budowlaną podobnie, ale jako \textit {zabudownąa działkę grutu(...)} -z czego wynika, że definicja nie jest spójna. Na potrzeby tej pracy pojęcie "działka budowlana" będzie oznaczała dzałkę niezabudowana, przeznazoną do zabudowy, w mysl pierwszej definicji.
Warto również zaznaczyć, że nawet jeśli w ogłoszeniu o sprzedaży dana działka jest budowlana, warto zweryfikować w  MPZP -  czy jest to zgodne z prawda i na prawdę nadaje się pod zabudową. Dla wiekszych miast te plany dostepne są w internecie, jednak w różnych foamratach- interaktywnej mapy, mapy pdf, oraz przede wszystkim - jako tekst uchwały samorządu. 


W kontekście szacoawania wartości dokumenty źródłowe (z Dziennika Ustaw) mówią raczej o wyceniniu nieruchomości niż działki czy działki budowlanej szeczólnosci.

Kodeks cywilny  definiuje nieruchomości jako
\textit { części powierzchni ziemskiej stanowiące odrębny przedmiot własności (grunty), jak również budynki trwale z gruntem związane lub części takich budynków(...)} \cite{KC} TODOCzęść Ogólna- Mienie  Art. 46.§ 1.


\section{Definicja wyceny nieruchomości}
W polskim prawodastwie 
\textit{ Określanie wartości nieruchomości polega na określaniu wartości prawa własności lub innych praw do nieruchomości.} \cite{RRM} Rozdz 2 par 2
W rozporządzeniu Rady Ministrów wspomniane są "rodzaje metod i technik wyceny nieruchomości", które służyć mają okresleniu wartości nieruchomości. W tym rozumieniu "wycena" będzie więc określeniem wartości" 
Wyceną nieruchomości zajmuje się  .... w sytuacji takiej i takie... tworzy......
Przy wykonianiu takiej wyceny rzeczoznawca może- w myśl rozporządzenia- zastosować następujące podejścia:
\begin{itemize}
\item dochodowe - w tym podejściu zasadniczą rolę odgrywa znajomość dochodów uzyskiwanych lub możliwych do uzyskania z nieruchomości, z czynszów i nie tylko; do samej wyceny można zaś zastosować dwie metody -  inwestycyjną (jeśli możliwa jest dzierżawa albo uzyskiwanie czynszu - wtedy trzeba przeanalizować dla nich stawki rynkowe) alb zysków (przanalizować trzeba wówczas inne dochody uzyskiwane z działaności prowadzonej na podobnych nieruchomościach).

W przypadku obu tych metod można zaś zastosować dwie techniki - techniki kapitalizacji prostej (\textit{wartość nieruchomości określa się jako iloczyn stałego strumienia dochodu rocznego możliwego do uzyskania z wycenianej nieruchomości i współczynnika kapitalizacji lub iloraz strumie-
nia stałego dochodu rocznego i stopy kapitalizacji.}) albo techniki dyskontowania strumieni dochodów (zbyt zawiłe???), przy czym przy obu tych technikach nie są brane pod uwagę koszty kredytu, podatku dochodowego, amortyzacji i innych opłat.

TODO gołe echniki dyskontowania strumieni dochodów § 9.-§ 14 i komentarz - czy to możliwe przy naszej wycenie??


\item mieszane - stosuje się metodę pozostałościową, metodę kosztów likwidacji albo metodę wskaźników szacunkowych gruntów
\item porównawcze - z metodami 
\begin{itemize}
\item porównywania parami - wycenianą nieruchomośc porównuje się kolejno z podobnymi nieruchomościami, które były przedmiotem obrotu rynkowego
\item korygowania ceny średniej - gdzie wylicza się średnią cenę transakcyjną z przynajmniej kilkunastu podobnych nieruychomości, a nastęopnie koryguje o poszzególne cechy tych nieruchomości.
\item  analizy statystycznej
\end{itemize}
Podejście porównawcze będzie najbliższe zastosowanemu do przeprowadzenia wyceny w dalszej części pracy, gdzie danymi wejściowymi dla sieci neuronowej będą dane o  cenach i cechach nieruchomości zebrane z dostępnych w internecie ofert.
W metodzie porównawczej uwzględniane są ceny tnasakcyjne, do których dostęp mają osoby biorące bezpośredni udział w transakcjach kupna-sprzedaży oraz biura nieruchomości. Znane powinny być również warunki zawarcia tych tansakcji  oraz to, jak cechy danej nieruchomości wpłynęły na cenę transakcyjną. Szczególną ostrożność trzeba zaś przyłożyć do cen związanych z przetargami - jesli odbiegają o więcej niż 20 \% od cen rynkowych, nie powinno się uch uwzględniać, oraz do cen transakcji zawartych przy zajściu szczególnych okoliczności - np \textit{sprzedaż dokonaną w postępowaniu egzekucyjnym,
sprzedaż z bonifikatą, sprzedaż z odroczonym terminem zapłaty lub sprzedaż z odroczonym terminem wydania nierucho-
mości nabywcy.}
Ceny powszechnie dostepne to ceny podawane w ogłoszeniach o sprzedaży nieruchomości, które są cenami ....TODO rynkowymi.


\end {itemize}

Proces wyceny nieruchomości powinna poprzedzać analiza rynku, tj rozeznanie się w zakresie warunków transakcji, cen, stawek czynszów.


\chapter{Źródła danych wejściowych do modeli oraz ich charakterystyka}
Dane wejściowe zebrane zostały  z internetowych portali z ogłoszeniami o nieruchomościach z serwisów: domiporta, morizon, otodom, gratka.
Dane wejściowe zawieraja zawsze cenę, przy czym nie jest to cena transakcyjna - co warunkuje to, że wszelkie próby wyceny poprzez sieci neuronowe nie będa w stanie okteslić faktycznej wartości nieruchomości.
Dane wejściowe są również niejednorodne (TODO czy ejst takie określenie???)

\chapter{Zastosowane metody pozyskiwania danych oraz przygotowania danych wejściowych}
Zrobiłam to i owo
\chapter{Analiza statystyczna potencjalnych danych wejściowych do modeli wraz z ich wyborem}
Analiza bla bla bla.
Wybrane te i te.
\chapter{Projekt i implementacja narzędzia do wyceny działek budowlanych}
Zaimpelmentowałam to tak i siak.
\chapter{Wykonanie wycen działek budowlanych oraz analiza otrzymanych wyników}
Działki a i b i c wycenione zostały tak i siak
\chapter{Podsumowanie i wnioski}

-------------------------------------
\chapter{Pozostałość ze starej pracy}

\section{Historia}

Rozdziały \ref{sectionSteganografiaWObiektachMultimedialnych} oraz 
\ref{chapterSteganografiaWRuchuTCPIP} opisują nowoczesne podejście do 
steganografii wykorzystujące współczesne kanały informacyjne. 
\section{Schemat komunikacji steganograficznej}
\label{sectionSchematKomunikacjiSteganograficznej}
 Ma on pełen wgląd do przekazywanych 
informacji, więc może przechwycić wszelkie przekazywane tajemnice, a dodatkowo w 
razie podejrzeń może nie dopuścić do komunikacji\footnote{podejrzana informacja 
jest tu analogią do stosowania kryptografii przez więźniów}. W takim przypadku w 
celu przekazania ważnych informacji \tech{A} i \tech{B} muszą posłużyć się 
pewnego rodzaju podstępem. 
\begin{figure}[!htbp]
	\begin{center}
\centering
\includegraphics[scale=0.4]{\ImgPath/rys/schemat_komunikacji.png}
\end{center}
	\caption{Schemat komunikacji steganograficznej}
	\label{schematKomunikacji}
\end{figure}

Przedstawioną tak sytuację pokazuje rysunek 
\ref{schematKomunikacji}\footnote{sporządzony na podstawie 
\cite{schematKomunikacjiPrzypis}, rysunek 1, strona 3}. \tech{A} próbuje 
przesłać tajną informację \tech{E} do \tech{B}. Można tu wykorzystać metody kryptografii symetrycznej (ustalony 
klucz kryptograficzny \tech{K}) lub niesymetrycznej (klucz publiczny 
\tech{K}$_{pub}$ i klucz prywatny \tech{K}$_{pryw}$).

\section{Metody tworzenia steganografii oraz rodzaje ukrytych kanałów}
 W większości przypadków występujących w rzeczywistych sieciach i 
systemach, numery wygenerowane przy pomocy \tech{Shushi} nie byłyby rozróżnialne 
od numerów wygenerowanych przez stos sieciowy systemu.

\begin{tabular}{c|cc}
pierwsza kolumna & druga & trzecia \\ \hline
1 & 2 & 3 \\
a & b & c \\
\end{tabular} 

\begin{equation}
 E = m c^2 \label{einstein}
\end{equation}



\appendix
\chapter{Porównanie numerów ISN jądra Linux i modułu Shushi}
\begin{figure}[!htbp]
	\begin{center}
\centering
\includegraphics[scale=0.21]{\ImgPath/rys/IPPortConstData.pdf}
\end{center}
	\caption{Numery ISN wygenerowane przez jądro oraz \tech{Shushi}, stałe 
numery IP oraz porty TCP, stałe dane dla \tech{Shushi}, serie po około 2800 
próbek.}
	\label{IPPortConstData}
\end{figure}

\begin{figure}[!htbp]
	\begin{center}
\centering
\includegraphics[scale=0.21]{\ImgPath/rys/IPPortConstDataDiff.pdf}
\end{center}
	\caption{Różnice pomiędzy kolejnymi numerami ISN wygenerowanymi przez 
jądro oraz \tech{Shushi}, stałe numery IP oraz porty TCP, stałe dane dla 
\tech{Shushi}, serie po około 60000 próbek.}
	\label{IPPortConstDataDiff}
\end{figure}


\begin{thebibliography}{99}
\addcontentsline{toc}{chapter}{Bibliografia}
\bibitem{KC}{Kodeks Cywilny https://isap.sejm.gov.pl/isap.nsf/download.xsp/WDU19640160093/U/D19640093Lj.pdf dost 11.12.2021}
\bibitem{OPRM}{OBWIESZCZENIE PREZESA RADY MINISTRÓW z dnia 3 marca 2021 r. w sprawie ogłoszenia jednolitego tekstu rozporządzenia Rady Ministrów w sprawie wyceny nieruchomości i sporządzania operatu szacunkowego; https://isap.sejm.gov.pl/isap.nsf/download.xsp/WDU20210000555/O/D20210555.pdf }
\bibitem{UoGN}{Ustawa z dnia 21sierpnia 1997r.o gospodarce nieruchomościami; http://isap.sejm.gov.pl/isap.nsf/download.xsp/WDU20200001990/U/D20201990Lj.pdf}
\bibitem{UoP}{U S T A W Az dnia 27marca 2003r.o planowaniu izagospodarowaniu przestrzennym; http://isap.sejm.gov.pl/isap.nsf/download.xsp/WDU20210000741/U/D20210741Lj.pdf}
\bibitem{OMI}{O B W I E S ZC ZE N I E  M I N I S T R A   I N W E S T Y C J I   I   R O ZW O J U1)z dnia 8 kwietnia 2019 r. w sprawie ogłoszenia jednolitego tekstu rozporządzenia Ministra Infrastruktury w sprawie warunków technicznych, jakim powinny odpowiadać budynki i ich usytuowanie}



\end{thebibliography}

\zakonczenie  % wklejenie recenzji i opinii

\end{document}
%+++ END +++
