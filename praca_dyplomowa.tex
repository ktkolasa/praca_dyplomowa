\documentclass[a4paper,12pt,twoside,openany]{report}

% Wzorzec pracy dyplomowej
% J. Starzynski (jstar@iem.pw.edu.pl) na podstawie pracy dyplomowej
% mgr. inż. Błażeja Wincenciaka
% Wersja 0.1 - 8 października 2016
%

\usepackage{polski}
\usepackage{helvet}
\usepackage[T1]{fontenc}
\usepackage{anyfontsize}
\usepackage[utf8]{inputenc}
\usepackage[pdftex]{graphicx}
\usepackage{tabularx}
\usepackage{array}
\usepackage[polish]{babel}
\usepackage{subfigure}
\usepackage{amsfonts}
\usepackage{verbatim}
\usepackage{indentfirst}
\usepackage[pdftex]{hyperref}


% rozmaite polecenia pomocnicze
% gdzie rysunki?
\newcommand{\ImgPath}{.}

% oznaczenie rzeczy do zrobienia/poprawienia
\newcommand{\TODO}{\textbf{TODO}}


% wyroznienie slow kluczowych
\newcommand{\tech}{\texttt}

% na oprawe (1.0cm - 0.7cm)*2 = 0.6cm
% na oprawe (1.1cm - 0.7cm)*2 = 0.8cm
%  oddsidemargin lewy margines na nieparzystych stronach
% evensidemargin lewy margines na parzystych stronach
\def\oprawa{1.05cm}
\addtolength{\oddsidemargin}{\oprawa}
\addtolength{\evensidemargin}{-\oprawa}

% table span multirows
\usepackage{multirow}
\usepackage{enumitem}	% enumitem.pdf
\setlist{listparindent=\parindent, parsep=\parskip} % potrzebuje enumitem

%%%%%%%%%%%%%%% Dodatkowe Pakiety %%%%%%%%%%%%%%%%%
\usepackage{prmag2017}   % definiuje komendy opieku,nrindeksu, rodzaj pracy, ...


%%%%%%%%%%%%%%% Strona Tytułowa %%%%%%%%%%%%%%%%%
% To trzeba wypelnic swoimi danymi
\title{Wycena działek budowlanych z wykorzystaniem uczenia maszynowego}

% autor
\author{Katarzyna Kolasa}
\nrindeksu{284919}

\opiekun{dr hab. inż.  Paweł Piotrowski}
\terminwykonania{1 lutego 2022} % data na oświadczeniu o samodzielności
\rok{2022}


% Podziekowanie - opcjonalne
%\podziekowania{\input{podziekowania.tex}}

\miasto{Warszawa}
\uczelnia{POLITECHNIKA WARSZAWSKA}
\wydzial{WYDZIAŁ ELEKTRYCZNY}
\instytut{INSTYTUT ELEKTROENERGETYKI}
%\instytut{INSTYTUT ELEKTROTECHNIKI TEORETYCZNEJ\linebreak[1] I~SYSTEMÓW INFORMACYJNO-POMIAROWYCH}
\zaklad{ZAKŁAD SIECI I SYSTEMÓW ELEKTROENERGETYCZNYCH}
% \zaklad{ZAKŁAD ELEKTROTECHNIKI TEORETYCZNEJ\linebreak[1] I~INFORMATYKI STOSOWANEJ}
%\kierunekstudiow{INFORMATYKA}

% domyslnie praca jest inzynierska, ale po odkomentowaniu ponizszej linii zrobi sie magisterska
%\pracamagisterska
%%% koniec od P.W

\opinie{%
  \input{opiniaopiekuna.tex}
  \newpage
  \input{recenzja.tex}
}

\streszczenia{
  \input{streszczenia.tex}
}

\begin{document}
\maketitle

\chapter{Wstęp do problemu wyceny działek budowlanych}

\section{Wprowadzenie}
Na początku przyjrzymy się definicjom, by doprecyzować temat niniejszej pracy, i sprawdzić, co można z nich zaczerpnąć na potrzeby tej pracy.

Jako źródł posłużą polskie akty prawne, które definiują pojęcia działki, działki budowlanej, a także nieruchomości. Pojęcie nieruchomości jest najbardziej ogóle, jednak istotne dla dalszych definicji, dlatego zostanie omówione na początku.


\section{Definicja działki budowlanej}
 Polski Kodeks Cywilny definiuje nieruchomości jako
\textit { części powierzchni ziemskiej stanowiące odrębny przedmiot własności (grunty), jak również budynki trwale z gruntem związane lub części takich budynków(...)} \cite{KC}
Ta definicja jest dosyć ogólna, ale wskazuje na zagadnienie własności, które może się okaząć później istotne.\\
Rodzaje działek gruntowych to: działki rolne, budowlane, siedliskowe, inwestycyjne, rekreacyjne, leśne. 


Pojęcie działki budowlanej doprecyzowane jest w kilku aktach prawnych, np w Ustawie o planowaniu i zagospodarowaniu przestrzennym, która definiuje ją jako \textit {nieruchomość gruntową lub działkę gruntu, której wielkość, cechy geometryczne, dostęp do drogi publicznej oraz wyposażenie w urządzenia infrastruktury technicznej spełniają wymogi realizacji obiektów budowlanych wynikające z odrębnych przepisów i aktów prawa miejscowego}
\cite{Uopizp}
Z tej definicji wynika, że działkę budowlaną można zabudować. Wskazuje równiez na kształ, dojazd, infrastrukturę jako potencjalne parametry przedmiotu pracy.

Tymaczesm Ustawa o gospodarce nieruchomościami  definiuje działkę budowlaną podobnie, ale jako \textit {zabudownąa działkę grutu(...)}\cite{Uogn} -z czego wynika, że definicja nie jest spójna. 
Na potrzeby tej pracy działkę budowlaną będziemy postrzegać bardzije w pierwszym ujęciu -jako teren, na którym prawo dopuszcza postawienie nieruchomości. Zdecydowanie zabudowanie działki nie będzie jednak warunkiem koniecznym. W rozszerzonej definicji jest to nie tylko teren, który jest przeznaczony do postawianie na nim nieruchomości, a także taki, który jest lub już był zabudowany.


Założeniem tej pracy jest skupienie się na działkach, czyli określenie wartości samego gruntu. W dalszej części pracy wskazane będzie, w jakim stoipniu możliwe jest rozdzielenie ofert sprzedaży działek zabudowanych i niezabudowanuch.


Co się tyczy dalszych warunków, jakie spełniać powinna działka budowlana, przeczytamy również: 


 \textit {Działka budowlana przewidziana pod zabudowę budynkami przeznaczonymi na pobyt ludzi powinna mieć zapewnioną możliwość przyłączenia uzbrojenia działki lub bezpośrednio budynku do sieci wodociągowej, kanalizacyjnej, elektroenergetycznej i ciepłowniczej, a dla budynków wymienionych w § 56 – także telekomunikacyjnej. 2. Za równorzędne z przyłączeniem do sieci elektroenergetycznej i ciepłowniczej uznaje się zapewnienie możliwości korzystania  z indywidualnych  źródeł  energii  elektrycznej  i ciepła,  odpowiadających  przepisom  odrębnym dotyczącym gospodarki energetycznej i ochrony Środowiska.}\cite{RMI}

Ten fragment wskazuje na potencjalne parametry danych wejściowych w dalszej części pracy: dostępność mediów i możliwość przyłączenia do sieci, co 


Warto również zaznaczyć, że nawet jeśli w ogłoszeniu o sprzedaży dana działka jest oznaczona jako budowlana, należy zweryfikować w  Miejscowym Planie Zagospodoarowania Przestrzennego, czy jest to zgodne z prawda i czy działka w istocie nadaje się pod zabudową. Dla wiekszych miast te plany dostepne są w internecie, jednak w różnych formatach- interaktywnej mapy, mapy pdf, oraz - przede wszystkim - jako tekst uchwały samorządu. 
Ponieważ na portalach ogłoszeniowych można podać przybliżoną lokalizację działki, ani właściciel portalu, ani użytkonik nie jest w istocie zweryfikować, czy podane przez ogłoszeniodawcę przeznaczenie działki jest zgodne z prawdą, z tego względu już na wstępnie trzeba liczyć się z potencjalnie źle oznaczonymi danymi.



\section{Definicja wyceny nieruchomości}
W tej sekcji sprawdzone zostanie, co formalnie oznacza pojęcie wyceny, i co można zaczerpnąć z formalnego opisu tego pojęcia. Na wstępnie zazbaczyć nalezy, że teksty źródłowe (akty prawne) traktują  o wycenianiu nieruchomości w ogóle, nie precyzując przypadku działki bądź działki budowlanej w szczególności.


W Dzienniku Ustaw czytamy, że \textit{określanie wartości nieruchomości następuje przy zastosowaniu poszczególnych podejść, metod, technik wyceny} \cite{OPRM}. W takim ujęciu wycena służy określaniu wartości nieruchomości, a \textit{wycena} i \textit{określenie wartości} nie są wyrażaniami synonimicznymi, ale na tyle bliskoznaczymi, że na potrzeby tej pracy w dalszej części będą używane zamiennie.
Rozporządzenie precyzuje też szereg środków, które umożliwiają zealizację zadania, jakim jest wycenam czyli \textit {określenie wartości prawa własności lub innych praw do nieruchomości}.

Wyceny nieruchomości dokonuje rzeczoznawca majątkowy, który tworzy  operat szacunkowy. By osiągnąć ten cel, może zastosować podejście porównawcze, dochodowe, mieszane oraz kosztowe. Każde podejście jest realizowane za pomocą odrębnych metod, z którymi z kolei wiążą się różne techniki.
Wyróżnione zostają następujące podejścia:
\begin{itemize}

\item dochodowe - w tym podejściu zasadniczą rolę odgrywa znajomość dochodów uzyskiwanych lub możliwych do uzyskania z nieruchomości, z czynszów i nie tylko; do samej wyceny można zaś zastosować dwie metody -  metodę inwestycyjną, jeśli możliwa jest dzierżawa albo uzyskiwanie czynszu - wtedy trzeba przeanalizować dla nich stawki rynkowe, albo metodę zysków - przanalizować trzeba wówczas inne dochody uzyskiwane z działaności prowadzonej na podobnych nieruchomościach
Obie metody realizować można przy uzyciu techniki kapitalizacji prostej albo dyskontowania strumieni dochodów, jednak nie będzie zasadnym zagłębianie się w nie. To podejście nie będzie miało zastosowania w dalszej częsci pracy ponieważ nie są dostępne informacje dotyczące dochodów z nieruchomości.



\item mieszane - stosuje się metodę pozostałościową, metodę kosztów likwidacji albo metodę wskaźników szacunkowych gruntów

\begin{itemize}
\item metoda pozostałościowa - stosuje się ją, gdy nie można zastosować podejścia porównawczego ani dochodowego; oraz jeśli na nieruchomości przeprowadzone będą jakiekolwiek roboty budowlane (związane z rozbudową, remontem, przebudową itp)
\item metoda kosztów likwidacji - jest stosowana, gdy składowa część nieruchomości jest przeznaczona do likwidacji
\item metoda wskaźnikóœ szacunkowych gruntów - dotyczy gruntów leśnych i rolnych, więc nie będących przedmiotem tej pracy
\end{itemize}

\item kosztowe - trzecie podejście do wyceny nieruchomości jest realizowane poprzez metody kosztów odtworzenia  lub zastąpienia,; co oznacza, że określa się koszty ottworzenia (lub zastąpienia)  części skłądowych gruntu częściami o takiej samej funkcji; podejście to realizowane jest poprzez 3 różne techniki; wartym odnotowania jest, że przy tym podejściu może wystaąpić sytuacjia, gdy wartość nieruchomości - ze względu na wysoki koszt przywrócenia jej do stanu umożliwiającego użytkowanie zgodnie z przeznaczniem- będzie liczbą ujemną.

\item porównawcze - jest czwartym i ostatnim podejściem, z którym wiążą się metody: 
\begin{itemize}
\item porównywania parami - wycenianą nieruchomośc porównuje się kolejno z podobnymi nieruchomościami, które były przedmiotem obrotu rynkowego
\item korygowania ceny średniej - gdzie wylicza się średnią cenę transakcyjną z przynajmniej kilkunastu podobnych nieruychomości, a nastęopnie koryguje o poszczególne cechy nieruchomości.
\item  analizy statystycznej - ustawa nie przybliża jednak konkretnych metod.
\end{itemize}


Podejście porównawcze opisane powyżej jest najbliższe zastosowanemu do przeprowadzenia wyceny w dalszej części pracy, gdzie danymi wejściowymi dla sieci neuronowej są dane o  cenach i cechach nieruchomości zebrane z dostępnych w internecie ofert. 
W metodzie porównawczej uwzględniane są ceny tnasakcyjne, do których dostęp mają osoby biorące bezpośredni udział w transakcjach kupna-sprzedaży oraz biura nieruchomości. Znane powinny być również warunki zawarcia tych tansakcji  oraz to, jak cechy danej nieruchomości wpłynęły na cenę transakcyjną. Szczególną ostrożność trzeba zaś przyłożyć do cen związanych z przetargami - jesli odbiegają o więcej niż 20 \% od cen rynkowych, nie powinno się uch uwzględniać, oraz do cen transakcji zawartych przy zajściu szczególnych okoliczności - np \textit{sprzedaż dokonaną w postępowaniu egzekucyjnym,sprzedaż z bonifikatą, sprzedaż z odroczonym terminem zapłaty lub sprzedaż z odroczonym terminem wydania nierucho-
mości nabywcy.}
Ceny powszechnie dostepne to ceny podawane w ogłoszeniach o sprzedaży nieruchomości, które są cenami rynkowymi, nie koniecznie są jednak cenami transakcyjnymi.

\end {itemize}

\section{Podsumowanie}

Z powyższch rozważań wynika, że 
Problem wyceny dziełek budowlanych związany jest z zagadnieniem wyceny nieruchomości w ogóle. 
Na poziomie portalu z ogłoszeniami o spżedaży działek nie można zweryfikowac, jakie są faktyczne warunki zabudowy gruntu.
Ponadtwo, działka pasująca do definicji działki budowlanej może być zabudowana lub nie.
Formalnie -do celów urzędowych - wyceną działki budowlanej zajmuje się rzeczoznawca majątkowy. Dokonuje wyceny na podstawie znajomości rynku, a szczególnie cen transakcyjnych, które nie są informacjami ogólnodostepnymi. Rzeczoznawca ma do dyspozycji szereg metod i technik, jednak nie są one opisane na tyle szczegółowo, by można było z nich czerpać na potrzeby tej pracy.
Uczenie sieci neuronowej wydaje się podobne do podejścia porównawczego.




\chapter{Źródła danych wejściowych do modeli oraz ich charakterystyka}

Pierwszym krokiem było zebranie danych o miastach w Polsce.
Zródłem byłą strona ..... . Z której zebrane zostały informacje w formacie ['Warszawa', 'powiat Warszawa', 'mazowieckie', '1 790 658', '517,2 km²']
i zapisane do jednego pliku, w porzadku od miasta z największą liczbą ludności do miasta z najmniejszą liczbą ludnosci (['Koźminek', 'powiat kaliski', 'wielkopolskie', '1 978', '-']).
Wszystkich zapisanych miast jest 951.
bla


Dane wejściowe zebrane zostały  z internetowych portali z ogłoszeniami o nieruchomościach z serwisów: domiporta, morizon, otodom, gratka.
Dane wejściowe zawieraja zawsze cenę, przy czym nie jest to cena transakcyjna - co warunkuje to, że wszelkie próby wyceny poprzez sieci neuronowe nie będa w stanie okteslić faktycznej wartości nieruchomości.
Dane wejściowe są również niejednorodne (TODO czy ejst takie określenie???)


- jednak na portalach, z których pochodzą dane, nie ma dostępnych narzędzi, by odfiltrować oferty w których jako działki budowlane zabudowane oraz niezabudowane nie są rozdzielone- co wiązać się może z zawyżoną ceną w stosunku do grutnów niezabudowanych o podobnych paramtrach.
Oferty sprzedaży domów, magazynów itp są osobną kategorią


\chapter{Zastosowane metody pozyskiwania danych oraz przygotowania danych wejściowych}
Zrobiłam to i owo
\chapter{Analiza statystyczna potencjalnych danych wejściowych do modeli wraz z ich wyborem}
Analiza bla bla bla.
Wybrane te i te.
\chapter{Projekt i implementacja narzędzia do wyceny działek budowlanych}
Zaimpelmentowałam to tak i siak.
\chapter{Wykonanie wycen działek budowlanych oraz analiza otrzymanych wyników}
Działki a i b i c wycenione zostały tak i siak
\chapter{Podsumowanie i wnioski}

-------------------------------------
\chapter{Pozostałość ze starej pracy}

\section{Historia}

Rozdziały \ref{sectionSteganografiaWObiektachMultimedialnych} oraz 
\ref{chapterSteganografiaWRuchuTCPIP} opisują nowoczesne podejście do 
steganografii wykorzystujące współczesne kanały informacyjne. 
\section{Schemat komunikacji steganograficznej}
\label{sectionSchematKomunikacjiSteganograficznej}
 Ma on pełen wgląd do przekazywanych 
informacji, więc może przechwycić wszelkie przekazywane tajemnice, a dodatkowo w 
razie podejrzeń może nie dopuścić do komunikacji\footnote{podejrzana informacja 
jest tu analogią do stosowania kryptografii przez więźniów}. W takim przypadku w 
celu przekazania ważnych informacji \tech{A} i \tech{B} muszą posłużyć się 
pewnego rodzaju podstępem. 
\begin{figure}[!htbp]
	\begin{center}
\centering
\includegraphics[scale=0.4]{\ImgPath/rys/schemat_komunikacji.png}
\end{center}
	\caption{Schemat komunikacji steganograficznej}
	\label{schematKomunikacji}
\end{figure}

Przedstawioną tak sytuację pokazuje rysunek 
\ref{schematKomunikacji}\footnote{sporządzony na podstawie 
\cite{schematKomunikacjiPrzypis}, rysunek 1, strona 3}. \tech{A} próbuje 
przesłać tajną informację \tech{E} do \tech{B}. Można tu wykorzystać metody kryptografii symetrycznej (ustalony 
klucz kryptograficzny \tech{K}) lub niesymetrycznej (klucz publiczny 
\tech{K}$_{pub}$ i klucz prywatny \tech{K}$_{pryw}$).

\section{Metody tworzenia steganografii oraz rodzaje ukrytych kanałów}
 W większości przypadków występujących w rzeczywistych sieciach i 
systemach, numery wygenerowane przy pomocy \tech{Shushi} nie byłyby rozróżnialne 
od numerów wygenerowanych przez stos sieciowy systemu.

\begin{tabular}{c|cc}
pierwsza kolumna & druga & trzecia \\ \hline
1 & 2 & 3 \\
a & b & c \\
\end{tabular} 

\begin{equation}
 E = m c^2 \label{einstein}
\end{equation}



\appendix
\chapter{Porównanie numerów ISN jądra Linux i modułu Shushi}
\begin{figure}[!htbp]
	\begin{center}
\centering
\includegraphics[scale=0.21]{\ImgPath/rys/IPPortConstData.pdf}
\end{center}
	\caption{Numery ISN wygenerowane przez jądro oraz \tech{Shushi}, stałe 
numery IP oraz porty TCP, stałe dane dla \tech{Shushi}, serie po około 2800 
próbek.}
	\label{IPPortConstData}
\end{figure}

\begin{figure}[!htbp]
	\begin{center}
\centering
\includegraphics[scale=0.21]{\ImgPath/rys/IPPortConstDataDiff.pdf}
\end{center}
	\caption{Różnice pomiędzy kolejnymi numerami ISN wygenerowanymi przez 
jądro oraz \tech{Shushi}, stałe numery IP oraz porty TCP, stałe dane dla 
\tech{Shushi}, serie po około 60000 próbek.}
	\label{IPPortConstDataDiff}
\end{figure}


\begin{thebibliography}{99}
\addcontentsline{toc}{chapter}{Bibliografia}

\bibitem{KC}{Kodeks Cywilny,  Część Ogólna- Mienie  Art. 46.§ 1. https://isap.sejm.gov.pl/isap.nsf/download.xsp/WDU19640160093/U/D19640093Lj.pdf dost 27.09.2022}
\bibitem{Uopizp} {Dz. U. 2003 Nr 80 poz. 717, Ustawa z dnia 27 marca 2003 r. o planowaniu i zagospodarowaniu przestrzennym, Art. 2 pkt 12  https://isap.sejm.gov.pl/isap.nsf/download.xsp/WDU20030800717/U/D20030717Lj.pdf dost 27.09.2022}
\bibitem {Uogn}{ Dz.U. 1997 nr 115 poz. 741, Art. 4 pkt 3a,  Ustawa z dnia 21 sierpnia 1997 r. o gospodarce nieruchomościami, https://isap.sejm.gov.pl/isap.nsf/download.xsp/WDU19971150741/U/D19970741Lj.pdf, dost 27.09.2022}
\bibitem{RMI}{Dz.U. 2002 nr 75 poz. 690, Rozdz. 5, § 26. 1.Rozporządzenie Ministra Infrastruktury z dnia 12 kwietnia 2002 r. w sprawie warunków technicznych, jakim powinny odpowiadać budynki i ich usytuowanie, https://isap.sejm.gov.pl/isap.nsf/download.xsp/WDU20020750690/O/D20020690.pdf, dost 27.09.2022 }

\bibitem{OPRM}{Dz.U. 2021 poz. 555, Obwieszczenie Prezesa Rady Ministrów z dnia 3 marca 2021 r. w sprawie ogłoszenia jednolitego tekstu rozporządzenia Rady Ministrów w sprawie wyceny nieruchomości i sporządzania operatu szacunkowego, Rozdz.2 § 3.1, https://isap.sejm.gov.pl/isap.nsf/download.xsp/WDU20210000555/O/D20210555.pdf dost 27.09.2022}

\bibitem{UoGN}{Ustawa z dnia 21sierpnia 1997r.o gospodarce nieruchomościami; http://isap.sejm.gov.pl/isap.nsf/download.xsp/WDU20200001990/U/D20201990Lj.pdf}
\bibitem{UoP}{U S T A W Az dnia 27marca 2003r.o planowaniu izagospodarowaniu przestrzennym; http://isap.sejm.gov.pl/isap.nsf/download.xsp/WDU20210000741/U/D20210741Lj.pdf}




\end{thebibliography}

\zakonczenie  % wklejenie recenzji i opinii

\end{document}
%+++ END +++
